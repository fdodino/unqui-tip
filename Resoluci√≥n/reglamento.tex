
\anexo{
TECNICATURA UNIVERSITARIA EN PROGRAMACIÓN INFORMÁTICA \\
REGLAMENTO DE TRABAJO DE INSERCIÓN PROFESIONAL \\
DEPARTAMENTO DE CIENCIA Y TECNOLOGÍA
}

\capitulo{DEFINICIÓN, OBJETIVOS Y RECONOCIMIENTO}  

\articulo El Trabajo de Inserción Profesional (TIP) es una instancia de
carácter integrador que debe realizar y presentar todo alumno de la carrera
Tecnicatura Universitaria en Programación Informática para obtener el título
correspondiente.

\articulo El TIP tiene como objetivos:
\referencia{Objetivos}
\begin{itemize}
 \item Desarrollar e integrar los conocimientos y formación adquiridos a lo
 largo de la carrera.
 \item Promover el uso de la creatividad e iniciativa para afrontar con
 eficiencia el trabajo proyectado.
 \item Asegurar el empleo de una correcta metodología y criterio
 profesional para el logro del plan propuesto.
\end{itemize}

\articulo Los créditos y horas que se asignen al TIP serán los
correspondientes al plan de estudios de la carrera vigente al momento de su
realización

\capitulo{CARACTERÍSTICAS, DURACIÓN Y ÁMBITO DE REALIZACIÓN DEL TRABAJO}

\articulo El TIP se podrá realizar en el ámbito de la Universidad Nacional de
Quilmes, en otras Universidades, en empresas, en Institutos de Investigación y
Desarrollo u otros organismos estatales o privados. 

\articulo El TIP será individual o en un grupo de a lo sumo dos estudiantes

\articulo La complejidad del TIP deberá permitir su realización en un
tiempo comprendido entre 100 (cien) y 180 (ciento ochenta) horas por estudiante de
acuerdo con el plan de estudios vigente. En los casos de trabajos compartidos, el
director o profesor a cargo debe certificar la equidad en la dedicación empleada
y la concreción de los objetivos mencionados en el \artObjetivos\ por cada uno de los
integrantes. \referencia{Horas}

\articulo Para presentar la propuesta de realización del TIP, el o
los estudiantes deberán acreditar un 80\% del total de créditos necesarios para
acceder al título de Técnico. \cambio{Se mueve la condición para la defensa del
TIP (una asignatura pendiente) porque cambia el concepto de defensa para el
modo cursada}
\referencia{CondInscripcion}

\articulo El tema deberá estar dentro de la órbita de incumbencia del título de técnico y
podrá ser autocontenido o parte integrante de un desarrollo más complejo para ser
eventualmente continuado o completado por otro u otros TIPs.

\articulo Tanto el software como la documentación desarrollada como producto
directo del TIP deben poseer licencias libres. Podrá admitirse excepciones a
este artículo mediante debida justificación.
\nuevo{Se pide abiertamente software y documentación libre}

\articulo El TIP podrá tener alguna de las siguientes modalidades:
\begin{enumerate}
\item Trabajo de investigación teórica o de teoría aplicada.
\item Pasantía calificada en un sector relacionado al software en una
empresa u organización estatal o privada.
\item Proyecto de desarrollo de un producto de software, a través de una
\modoTutorado.
\item Proyecto de desarrollo de un producto de software, en modalidad \modoCursada.
\end{enumerate}
\nuevo{agrega otra modalidad}

El \capTradicional rige únicamente para las tres primeras modalidades
El \capCursada rige únicamente para la cuarta modalidad.
\nuevo{separa la reglamentación de las modalidades tradicionales y la nueva
modalidad}

\capitulo{ REGLAMENTACIÓN DEL TRABAJO DE INSERCIÓN PROFESIONAL MEDIANTE TUTORÍA LIBRE, INVESTIGACIÓN O PRÁCTICA PROFESIONAL}
\referencia{Tradicional}
\nuevo{se agrupa en un capítulo todos los antiguos capítulos que hacen
referencia a los modos tradicionales. Los antiguos capítulos son renombrados
como secciones}

%TODO llevarme esto al capitulo tradicional
%Para la defensa del TIP, no podrán adeudar más de
%una asignatura

\articulo El proceso de desarrollo del TIP se articulará en diversas etapas:
\begin{itemize}
 \item Presentación de la propuesta.
 \item Desarrollo del trabajo propiamente dicho.
 \item Presentación final.
 \item Evaluación y defensa.
\end{itemize}

\articulo La duración del TIP, desde la aprobación de la presentación hasta
su presentación final, está prevista en 6 (seis) meses. 
\referencia{Tiempo}

\seccion{ DIRECCIÓN DEL TRABAJO DE INSERCIÓN PROFESIONAL}

\articulo El TIP deberá contar con una dirección. La misma será llevada a
cabo por un Director y opcionalmente un Codirector si lo justifica el tema, plan de
trabajo u otras condiciones.

\articulo El Director del trabajo, y el Codirector, si existe,
deberán cumplir con al menos uno de los siguientes requisitos:
\referencia{CondDirector}
\begin{enumerate}
\item Ser profesor ordinario o interino en asignaturas de la Carrera, con al
menos dos cuatrimestres de docencia en los últimos dos años.
\item Ser profesor de otras carreras de la UNQ o de otras Universidades en
asignaturas afines a carreras de informática con al menos dos años de
antigüedad.
\item Ser investigador reconocido en temáticas afines al tema del trabajo.
\item Ser profesional de la industria informática con reconocida trayectoria en
temas afines al tema del trabajo.
\end{enumerate}
\nota{eliminar la antigüedad requerida?}

\articulo Cuando el director sea externo a la UNQ y no hubiera figura de
Codirector interno, deberá designarse un profesor ordinario o interino como tutor
académico de la institución. \nota{Esto no lo cambié, pero lo que pasa es que
siempre exigimos un codirector interno y no usamos la figura de tutor académico}

\articulo El Director y Codirector sólo podrán dirigir y/o codirigir hasta 3
(tres) trabajos simultáneamente. 

\articulo Son funciones del Director y del Codirector si lo hubiere:
\begin{itemize}
 \item Decidir con el estudiante el tema y plan del TIP verificando que la
 propuesta puede desarrollarse en los tiempos establecidos en el
 \artHoras.
 \item Supervisar el efectivo cumplimiento y dedicación del estudiante en
 sus actividades, estableciendo un método de seguimiento y
 manteniendo reuniones periódicas con el estudiante.
 \item Coordinar con el Director de Carrera el uso de los recursos de la
 UNQ verificando su disponibilidad.
 \item Controlar que se cumpla con el alcance del TIP propuesto en la
 presentación correspondiente.
 \item Ser plenamente responsable de todo lo presentado por el
 estudiante en relación al TIP.
 \item Informar periódicamente a la Dirección de Carrera el estado de
 avance del TIP.
 \item Actuar como nexo entre el jurado evaluador y el estudiante.
 \item Oficiar como presentador del alumno y el TIP en la defensa oral del
 mismo.
 \item Participar activamente en la detección y solución de los problemas
 que pudieran surgir durante el desarrollo del trabajo.
\end{itemize}


\seccion{PROPUESTA INICIAL}

\articulo Para comenzar el TIP, el estudiante deberá solicitar la aceptación
de su propuesta de Trabajo de Inserción Profesional mediante una presentación (ver
\anexoPresentacionPlan) firmada por el Director propuesto (y Codirector si lo
hubiere) y el o los estudiantes. En la misma deberá constar:

\begin{enumerate}[a.]
\item Tema del TIP.
\item Justificación e importancia del tema seleccionado.
\item Objetivos del trabajo.
\item Lugar donde se realizará el trabajo.
\item Plan de trabajo propuesto y su cronograma.
\item Licencias con las cuáles se distribuirán el código fuente y la
documentación.
Justificación en caso de licencia privativa \nuevo{ }
\item Fuentes iniciales de información y referencia.
\item Porcentaje de asignaturas aprobadas por cada uno de los estudiantes.
\item Foja académica de cada uno de los estudiantes.
\item Currículum vítae del director y codirector, si existe.
\end{enumerate}

\articulo Las propuestas de Trabajo de Inserción Profesional podrán ser
presentadas entre el 1 de Marzo y el 30 de Noviembre de cada año, excluyendo el
receso invernal.

\articulo La propuesta debidamente encarpetada debe ser entregada al
Director de la Carrera, quién podrá consultar a profesores de asignaturas afines al
tema de trabajo, licencias y al Director o Codirector propuesto, y luego, en un
plazo no mayor a 15 días de recibida la propuesta, elevará un dictamen al Consejo departamental,
quién decidirá sobre su aprobación. Tanto el Director de la Carrera como la Comisión de Investigación
Científica y Tecnológica podrán solicitar los ajustes que consideren necesarios para la
admisión de los TIPs. 
\nuevo{se agrega las licencias}

\articulo Si el trabajo se realiza en un ámbito externo a la UNQ, deberá
contarse con el aval de la empresa o institución correspondiente. El Director de la
carrera elevará, de acuerdo a la normativa vigente en el Departamento, para los fines
de la cobertura por ART

\begin{itemize}
 \item una nota indicando apellido y nombre del/los alumno/s, DNI, fecha
 de nacimiento, actividad a desarrollar (en este caso, Trabajo de
 Inserción Profesional, con el tema y otros datos pertinentes) y lugar
 de desarrollo,
 \item certificación/aval del Director/Codirector externo (en caso de
 haberlo/s),
 \item certificación/Aval de Autoridad competente del ámbito externo (en
 caso de desarrollarse la actividad fuera de la UNQ).
\end{itemize}


\articulo El Consejo Departamental aprobará mediante resolución la
realización de los TIPs, especificando:
\begin{itemize}
 \item Apellido, nombre y número de legajo de los estudiantes.
 \item Fecha de inicio (que contará como el inicio de los plazos de
 realización del \artTiempo) y fecha estimada de finalización.
 \item Director (y Codirector o Tutor, si lo hubiere).
 \item Tema.
 \item Lugar de trabajo.
\end{itemize}

Se notificará debidamente al estudiante junto con las observaciones y/o cambios
propuestos para la aprobación del plan.

\seccion{SEGUIMIENTO DEL TRABAJO}

\articulo En caso de incumplimiento por parte del o de los estudiantes, el
Director deberá notificar por escrito al Director de Carrera y podrá solicitar su
desvinculación del trabajo. El Director de Carrera deberá expedirse en un plazo
máximo de 15 días sobre la procedencia o improcedencia de dicha solicitud. En caso
de considerar procedente el planteo interpuesto por el Director, el Director de Carrera
deberá comunicar tal situación al Consejo Departamental. Si se ha aceptado la
desvinculación el Director de Carrera deberá comunicar el cese del desarrollo del TIP
y el estudiante deberá cumplir con lo estipulado en el \artFinalizado.

\articulo En el caso de que el o los estudiantes consideren que el Director
o Codirector del trabajo incurre o incurren en incumplimiento de sus obligaciones.
deberá informar por nota al Director de la Carrera. El Director de Carrera deberá
expedirse en un plazo máximo de 15 días sobre la procedencia o improcedencia de
dicha solicitud. En caso de considerar procedente el planteo interpuesto por el
estudiante, el Director de Carrera deberá comunicar tal situación al Consejo
Departamental y presentar la propuesta de un nuevo Director o Codirector. 

\articulo \referencia{Prorrogas} El Director de la Carrera podrá autorizar una
prórroga de hasta 6 meses ante un pedido escrito y debidamente fundamentado del estudiante y avalado
por el Director del TIP (y el Codirector si lo hubiere). Excepcionalmente el Consejo
Departamental podrá otorgar una extensión no mayor a 6 meses adicionales ante
solicitud fundada presentada por el estudiante y avalada por el Director del TIP (y el
Codirector si lo hubiere) y el Director de la Carrera. 

\articulo El estudiante podrá solicitar al Director de la Carrera una
suspensión de plazos mediante nota debidamente justificada y avalada por el Director
del TIP (y el Codirector si lo hubiere). De serle concedida se interrumpirá la cuenta de
los tiempos establecidos en el \artTiempo (incluyendo las prórrogas mencionadas
en el \artProrrogas). El Director de la Carrera deberá informar al Consejo
Departamental acerca de las suspensiones de plazos. En el caso de una suspensión mayor a 3 (tres)
meses el estudiante deberá dejar disponible para otros usos todo material, insumo,
lugar físico o equipamiento provisto por la UNQ.

\articulo \referencia{Finalizado} El TIP se dará por finalizado si no se hubiese
efectuado la defensa transcurridos los plazos establecidos en el \artTiempo
(incluyendo las prórrogas mencionadas en el artículo \artProrrogas). En este
caso el estudiante deberá desocupar el lugar físico y devolver el equipamiento provisto por la UNQ y restituir todo material o insumo
utilizado en el proceso inconcluso.

\seccion{EVALUACIÓN}

\articulo La culminación del TIP implica su presentación en formatos digital y
 escrito y su defensa oral y pública.
 \nuevo{agrega el concepto de entrega digital}

\articulo Para la evaluación del TIP se formará
un Jurado evaluador, propuesto por el Director del trabajo y conformado por el
Director de la carrera.

\articulo El jurado constará de tres integrantes, al menos dos de los cuales
serán profesores pertenecientes a la UNQ y al menos uno de ellos a la Carrera de
Tecnicatura Universitaria en Programación Informática. El miembro restante podrá ser
un profesional de reconocida trayectoria y ser externo a la Universidad Nacional de
Quilmes. El Director y/o Codirector del TIP y el Director de la Carrera no podrán ser
miembros del Jurado. Todos los integrantes del Jurado deben cumplir con al menos
una de las condiciones requeridas en el \artCondDirector.

\articulo El o los estudiantes y el Director del TIP (y Codirector si lo
hubiere) podrán sugerir, por nota escrita al Director de la Carrera, potenciales
integrantes del jurado.

\articulo El o los estudiantes en conjunto con el Director del TIP (y
Codirector si lo hubiere) podrán recusar o pedir el cambio de Jurado con la
fundamentación adecuada mediante nota al Director de la Carrera quien la elevará al
Consejo Departamental.

\articulo En el caso de los jurados externos a la UNQ se deberá contar
con una aceptación de la tarea por parte de los mismos. En el caso de los jurados
internos de la UNQ podrán excusarse mediante justificación debidamente fundada. 

\seccion{ENTREGA DEL TRABAJO}
\cambio{Antes se llamaba PRESENTACIÓN DEL INFORME}

\articulo La presentación del trabajo será realizada en primera instancia en
formato digital. La misma debe ir acompañada de una nota escrita en la que se
propone los integrantes del jurado al Director de carrera, según el
\anexoEntregaDigitalTIP.
\cambio{Antes se pedía un formato escrito en esta instancia}

\articulo La entrega incluye la realización de un informe y el código fuente del
trabajo. El informe debe incluir en su cuerpo o en forma de anexos,
documentación de diseño de software e instrucciones de instalación y
configuración.
\nuevo{Antes se pedía solo el informe y no estaba explícito que debía
contener. Se solicita ahora además del informe la entrega del código fuente.
Además, que el informe incluya la documentación de diseño y el manual de
instalación y configuración. Se busca reemplazar el estilo de un informe que
tiene mucha prosa a un estilo con mayor documentación técnica.}
\nota{¿Ser más explícito acerca de lo que se espera del informe o no? ¿Pedir
explícitamente que explique el problema? ¿Pedir que incluya explicaciones de
cuestiones teóricas que entran en juego en el trabajo?}

\articulo En el caso de realizar el trabajo con licencias privativas, el director del trabajo 
propondrá, con el aval de la dirección de carrera,
un mecanismo por el cual el jurado tenga acceso al material licenciado para su
evaluación. 
\nuevo{es responsabilidad del director de trabajo que el jurado tenga acceso a
todo}

\articulo La presentación por escrito del informe TIP deberá ser realizada
teniendo en cuenta las recomendaciones que figuran en el \anexoPresentacionInforme, acompañada de una nota
según el modelo del \anexoPresentacionInformeEscrito. La misma debe realizarse
luego de la habilitación del jurado para la defensa oral, y antes de llevarse a
cabo la misma.
\cambio{aclaro que la presentación escrita es el informe. el trabajo completo
está digital}
\cambio{La entrega escrita es posterior al que jurado da el OK}

\articulo En primera instancia se remitirá al Jurado todo el material entregado
en formato electrónico para la elaboración del dictamen inicial. El Jurado tendrá un
plazo no superior a los 30 (treinta) días corridos a partir de la recepción de la versión
electrónica, para realizar una evaluación preliminar y comunicar al Director de la
Carrera su dictamen inicial, fundamentado y por escrito. El jurado podrá expedirse en
tres alternativas:

\begin{enumerate}[a.]
\item Habilitado para la defensa oral.
\item Devuelto con observaciones.
\item Rechazado con dictamen fundado.
\end{enumerate}

\cambio{Se entrega todo el material electrónico para la evaluación}
\referencia{DictamenInicial}

\articulo En el caso del \artDictamenInicial, inciso a., los estudiantes
estarán habilitados para la presentación oral. 

\articulo En el caso del \artDictamenInicial, inciso b., los estudiantes podrán rehacer
la presentación una única vez, con las consideraciones que haya sugerido el Jurado.
En función de las correcciones realizadas, el Jurado determinará un plazo de entre
uno y tres meses para la realización de las modificaciones y solicitar un nuevo
dictamen inicial.

\articulo En el caso del \artDictamenInicial, inciso c., los estudiantes entrarán en las
condiciones pautadas en el \artFinalizado. 

\seccion{DEFENSA ORAL}

\articulo Para proceder a la defensa oral, el o los alumnos podrán adeudar a lo
sumo 1 materia. \nota{no es nuevo, estaba en uno de los primeros artículos y lo
moví acá para que no conflictúe con la modalidad presencial}

\articulo Una vez habilitados para la defensa oral, los estudiantes
entregarán 2 (dos) ejemplares impresos de la versión definitiva a la Dirección de la
Carrera, encuadernados en formato A4 y debidamente numerados.

\articulo El o los estudiantes y el Director (y Codirector si lo hubiere)
coordinarán la fecha de defensa oral con los miembros del Jurado, dentro de los 30
días a partir de la entrega de la versión final.

\articulo La Dirección de la Carrera se hará cargo de las gestiones
necesarias que habiliten la defensa oral y pública en el día prefijado.

\articulo La defensa oral y pública constará de una exposición que incluirá
los aspectos más sobresalientes del trabajo realizado. Tendrá una duración mínima de
30 minutos y máxima de 45 minutos, y una sesión consecutiva de preguntas y
consideraciones de los miembros del tribunal.

\articulo En la defensa oral y pública el dictamen del jurado, con el criterio
de la mayoría simple, podrá resultar: 
\begin{enumerate}[a.]
\item Aprobado con dictamen fundado, con una nota numérica entre 4 y 10.
\item Rechazado con dictamen fundado, con una nota numérica entre 1 y 3.
\end{enumerate}

\nuevo{Agrego la nota numérica, es algo que ya se venía haciendo, pero no
quedaba en el dictamen, si no en las actas de alumnos.}

\capitulo{REGLAMENTACIÓN DEL TRABAJO DE INSERCIÓN PROFESIONAL MEDIANTE CURSADA PRESENCIAL}
\referencia{Cursada}
\nuevo{Todo el capítulo es nuevo}

\seccion{ RÉGIMEN DE CURSADA}

\articulo Los alumnos que opten por esta modalidad, deberán inscribirse a la
misma al comienzo del cuatrimestre, utilizando los mismos
mecanismos con los cuales se inscriben a las materias que forman parte de la currícula. 

\articulo Los requisitos para inscribirse a la cursada siguen la misma normativa
que para el resto de las otras modalidades de TIP, según el \artCondInscripcion.

\articulo La dirección de carrera, en conjunto con la secretaría académica,
establecerán un aula y un horario semanal en el cual se llevarán adelante las
tareas de desarrollo, seguimiento y supervisión del TIP. Se designarán 5 horas
semanales para tal actividad.

\articulo El inicio y fin del régimen de cursada se establecerá de acuerdo al
calendario académico vigente. \referencia{Regimen}

\seccion{ PROFESOR DEL TRABAJO DE INSERCIÓN PROFESIONAL}

\articulo El TIP deberá contar con una supervisión. La misma será llevada a
cabo por el \profesorTIP{}. 

\articulo Podrá ser nombrado \profesorTIP{} cualquier docente interino u
ordinario de la UNQ, quien debe poseer título de grado afín a la informática o
mérito equivalente.
\nota{quedó bien esto? nótese que para este cargo no se exige antigüedad como
en la dirección de TIP}

\articulo Son funciones del \profesorTIP{}:
\begin{itemize}
 \item Decidir el tema y plan del TIP verificando que la
 propuesta puede desarrollarse en los tiempos establecidos en el
 \artHoras y dentro del régimen de cursada establecido en el \artRegimen.
 \item Supervisar el efectivo cumplimiento y dedicación del estudiante en
 sus actividades, estableciendo un método de seguimiento y
 manteniendo reuniones periódicas con el o los estudiantes.
 \item Coordinar con el Director de Carrera el uso de los recursos de la
 UNQ verificando su disponibilidad.
 \item Controlar que se cumpla con el alcance del TIP propuesto en la
 presentación correspondiente.
 \item Evaluar al alumno en las presentaciones periódicas y final del TIP
 \item Participar activamente en la detección y solución de los problemas
 que pudieran surgir durante el desarrollo del trabajo.
 \item Cumplir con todos los requisitos exigidos por la universidad para los
 profesores a cargo de curso.
 \item Elevar a la dirección de carrera, al finalizar la cursada, las entregas
 finales aprobadas de todos los TIP realizados durante el curso.
\end{itemize}

\articulo El \profesorTIP{} podrá solicitar a la dirección de carrera que
se designe a un especialista como consultor técnico en el caso de que el tema
del proyecto lo requiera, cuya responsabilidad es asistir técnicamente al
\profesorTIP y al o los estudiantes.

\nota{hace falta poner los requisitos para ser consultor? y las supongo que no,
creo que queda claro que alcanza con ser un especialista}

\seccion{PROPUESTA INICIAL}

\articulo Para comenzar el TIP, el estudiante debe hacer la propuesta de desarrollar un producto de
software. En la misma deberá constar:

\begin{enumerate}[a.]
\item Tema del TIP.
\item Objetivos del trabajo.
\item Plan de trabajo propuesto y su cronograma, que debe seguir un esquema iterativo con entregas
incrementales.
\item Arquitectura propuesta.
\item Currículum vítae del o de los estudiantes
\end{enumerate}

\articulo Las propuestas de Trabajo de Inserción Profesional deben ser presentadas
dentro de las primeras tres semanas de la cursada, acompañada de una prueba de concepto
que sirva como estudio de factibilidad práctico.

\articulo La propuesta debe ser presentada dentro del horario de cursada previsto al \profesorTIP{}
quien podrá solicitar los ajustes que considere necesarios para la admisión de los TIP. 

\seccion{SEGUIMIENTO DEL TRABAJO}

\articulo Una vez establecidas las pautas de trabajo y los plazos, 
su cumplimiento es obligatorio para conservar la regularidad de la cursada y la
consiguiente aprobación del TIP.

\articulo En el caso de no poder asistir a cualquiera de las entregas presenciales, el estudiante 
debe informar por escrito en un plazo razonable el motivo de la inasistencia y proponer un nuevo 
horario al \profesorTIP{}.

\seccion{EVALUACIÓN}

\articulo La aprobación del TIP implica la aprobación de las entregas parciales
y la final, que consistirán en 
\begin{itemize}
 \item realizar una demostración del producto de software desarrollado
 \item presentar la documentación respaldatoria del trabajo en formato digital
 \item responder las preguntas que formule el \profesorTIP{}
\end{itemize}

\seccion{PRESENTACIÓN DE LAS ENTREGAS}

\articulo Cada entrega deberá constar de
\begin{itemize}
 \item un incremento o producto de software funcionando
 \item la documentación respaldatoria que puede presentarse formato digital
 \item y la presentación de la entrega al \profesorTIP{}
\end{itemize}

\nota{Puse que las entregas son en formato digital, eso no implica que puedan
además traer el papel impreso, pero quiero exigir la entrega digital porque eso
en más fácil de archivar}

\articulo La entrega final, luego de ser aprobada, deberá incluir dos ejemplares
debidamente encuadernados de toda la documentación respaldatoria del trabajo.

\seccion{PRESENTACIÓN ORAL}

\articulo Una vez que el \profesorTIP{} haya aprobado el TIP al o a los estudiantes, 
se hará una presentación formal, abierta y pública dentro del ámbito de la Universidad. 

\articulo Para realizar la presentación, el o los estudiantes podrán adeudar
hasta una materia cada uno.
\referencia{CondPresentacion}

\articulo En caso de cumplir con \artCondPresentacion al momento de la
aprobación de la entrega final, el o los estudiantes y el \profesorTIP{}
coordinarán la fecha de presentación oral y publica con el Director de la
Carrera dentro de los 30 días siguientes.

\articulo En caso de no cumplir con \artCondPresentacion al momento de la
aprobación de la entrega final. La presentación oral y pública quedará en
suspenso hasta que se cumpla con dicho artículo, luego de lo cual, los estudiantes coordinarán con el
Director de la Carrera la fecha de presentación oral dentro de los 30 días
siguientes.

\articulo Es condición necesaria la realización de la presentación oral para la
aprobación  del TIP.

\articulo La Dirección de la Carrera se hará cargo de las gestiones
necesarias que habiliten la presentación oral y pública en el día prefijado.

\articulo La presentación oral y pública constará de una exposición que incluirá
los aspectos más sobresalientes del trabajo realizado. Tendrá una duración mínima de
30 minutos y máxima de 45 minutos.

\capitulo{ARCHIVO Y DIFUSIÓN}

\articulo Una vez aprobado el TIP un ejemplar será destinado a la
Biblioteca general de la Universidad Nacional de Quilmes y otro quedará en la
Dirección de la Carrera. Para la documentación con formato digital, la Dirección de Carrera
proveerá los mecanismos que permitan almacenar la información para continuar futuros trabajos e investigaciones.
\cambio{Agrega la documentación digital}

\articulo Los eventuales aspectos legales referidos a la propiedad
intelectual que pueden derivar del desarrollo de los Trabajos de Inserción Profesional
serán resueltos de acuerdo a los reglamentos y resoluciones vigentes en la
Universidad Nacional de Quilmes.

\capitulo{VÍA DE EXCEPCIÓN}

\articulo Toda excepción a este reglamento deberá ser analizada en
primera instancia por el Director de Carrera y aprobada por el Consejo Departamental.

\newpage
\resetAnexosCounter
\anexo{NOTA MODELO del alumno para aprobación de la propuesta de trabajo de
inserción profesional}
\referencia{PresentacionPlan}

OBSERVACIÓN: los campos consignados como \textless...\textgreater deben ser reemplazados
por el contenido correspondiente. Aquellos que ofreciesen alternativas
 \textless.../...\textgreater deberán reemplazarse por la alternativa
 correspondiente.

\begin{flushright}
Bernal, \param{día} de \param{mes} de \param{año}\end{flushright}

Al Director de la Carrera de Tecnicatura Universitaria en Programación Informática del
Departamento de Ciencia y Tecnología de la
Universidad Nacional de Quilmes,\\
\param{Título, Nombre y Apellido del Director de Carrera}

\hrulefill

Asunto: Propuesta de trabajo de inserción profesional

\param{Me dirijo/Nos dirigimos} a Usted para presentar \param{mi/nuestra}
propuesta de Trabajo de Inserción Profesional de la carrera de Tecnicatura
Universitaria en Programación Informática. El tema del trabajo es: \param{Tema
del trabajo}, y será dirigido por \param{Título, Nombre y Apellido del Director
del Trabajo>} y \param{codirigido por: \param{Título, Nombre y Apellido del
Codirector del Trabajo} si corresponde}.

Adjunto con esta nota la justificación e importancia del tema
seleccionado, los objetivos del trabajo, el lugar donde se ha de realizar el
mismo, el plan de trabajo y su cronograma así como las fuentes iniciales de
información y referencia, según el reglamento vigente. \param{Poseo/Poseemos}
aprobadas el \param{Porcentaje de aprobadas}\% \param{y el \param{Porcentaje de
aprobadas del alumno 2}, si corresponde} de las materias de la carrera (se
\param{adjunta certificado analítico/adjuntan certificados analíticos}).

Sin otro particular \param{saluda/saludan} atentamente.\\ \\
\param{Firma, aclaración y Nro Legajo de alumno 1}\\
\param{Firma, aclaración y Nro Legajo de alumno 2, si corresponde}\\
\param{Firma y aclaración Director}\\
\param{Firma y aclaración Codirector, si corresponde}\\

\newpage
\anexo{PRESENTACIÓN ESCRITA DEL TRABAJO DE INSERCIÓN PROFESIONAL}
\referencia{PresentacionInforme}
Primera Parte: Forma.
\begin{enumerate}
\item Hojas A4, numeradas, letra tamaño 12pt.
\item Trabajo encuadernado o anillado.
\item Modelo de portada (centrado en la hoja).
\begin{center}
UNIVERSIDAD NACIONAL DE QUILMES\\
Departamento de Ciencia y Tecnología\\
Tecnicatura Universitaria en Programación Informática\\

\param{TITULO DEL TRABAJO}\\
\param{NOMBRE DEL ALUMNO O ALUMNOS}\\
\param{DIRECTOR (Y CODIRECTOR) DEL PROYECTO}\\
\param{FECHA}\\
\end{center}
\end{enumerate}

Segunda Parte: Contenido.
La estructura y contenido del trabajo de inserción profesional dependerán de la
modalidad del trabajo. En todos los casos de deberá incluir:
\begin{enumerate}[a.]
\item Resumen.
\item Índice.
\item Introducción e importancia del tema.
\item Objetivos.
\item Marco teórico.
\item Conclusiones.
\item Referencias bibliográficas.
\end{enumerate}

\newpage
\anexo{

NOTA MODELO para la presentación de la entrega del Trabajo de Inserción
Profesional}
\referencia{EntregaDigitalTIP}

OBSERVACIÓN: los campos consignados como \textless...\textgreater deben ser
reemplazados por el contenido correspondiente. Aquellos que ofreciesen
alternativas \textless.../...\textgreater deberán reemplazarse por la/s
alternativa/s correspondiente/s.

\begin{flushright}
Bernal, \param{día} de \param{mes} de \param{año}
\end{flushright}

Al Director de la Carrera de Tecnicatura Universitaria en Programación Informática del
Departamento de Ciencia y Tecnología de la
Universidad Nacional de Quilmes,

\param{Título, Nombre y Apellido del Director de Carrera}

\hrulefill

\param{Tengo/Tenemos} el agrado de \param{dirigirme/dirigirnos} a
usted para realizar la entrega digital del Trabajo
de Inserción Profesional titulado
\param{Título del trabajo de inserción profesional}, que fuera dirigido por \param{Título, Nombre y Apellido del Director del Trabajo}
\param{/y codirigido por \param{Título, Nombre
y Apellido del Codirector del Trabajo}, si corresponde}. La entrega incluye
\param{El informe del trabajo realizado /código fuente / documentación
de diseño y de uso / manual de instalación / otros documentos u archivos
pertinentes}

Se propone como posibles jurados titulares a \param{Título,
Nombre y Apellido del Jurado 1}, \param{Título, Nombre y Apellido del Jurado 2}
y \param{Título, Nombre y Apellido del Jurado 3} y como posibles jurados
suplentes a \param{Título, Nombre
y Apellido del Jurado 4} y \param{Título, Nombre y Apellido del Jurado 5}.
del jurado, y de su expedición sobre el trabajo. Cumplido esto, coordinaremos la fecha
de defensa oral.

\param{Quedo/Quedamos} a la espera de la designación oficial

Sin otro particular lo saluda atentamente\\ \\
\begin{flushright}
\param{Firma del alumno 1}\\
\param{Nombre del alumno 1}\\
\param{Legajo 1}\\
\param{Correo electrónico 1}\\
\param{/Firma del alumno 2, si corresponde}\\
\param{/Nombre del alumno 2, si corresponde}\\
\param{/Legajo 2, si corresponde}\\
\param{/Correo electrónico 2, si corresponde}\\
\param{Firma y aclaración Director}\\
\param{/Firma y aclaración Codirector, si corresponde}\\
\end{flushright}

\anexo{

NOTA MODELO para la presentación del informe final escrito del Trabajo de
Inserción Profesional}

OBSERVACIÓN: los campos consignados como \textless...\textgreater deben ser
reemplazados por el contenido correspondiente. Aquellos que ofreciesen
alternativas \textless.../...\textgreater deberán reemplazarse por la/s
alternativa/s correspondiente/s.


\referencia{PresentacionInformeEscrito}

\begin{flushright}
Bernal, \param{día} de \param{mes} de \param{año}
\end{flushright}

Al Director de la Carrera de Tecnicatura Universitaria en Programación Informática del
Departamento de Ciencia y Tecnología de la
Universidad Nacional de Quilmes,

\param{Título, Nombre y Apellido del Director de Carrera}

\hrulefill

\param{Tengo/Tenemos} el agrado de \param{dirigirme/dirigirnos} a
usted para hacerle llegar 2 (dos) copias del informe escrito del Trabajo de Inserción
Profesional titulado \param{Título del trabajo de inserción profesional}, que
fuera dirigido por \param{Título, Nombre y Apellido del Director del Trabajo}
\param{/y codirigido por \param{Título, Nombre
y Apellido del Codirector del Trabajo}, si corresponde}

Sin otro particular lo saluda atentamente\\ \\
\begin{flushright}
\param{Firma del alumno 1}\\
\param{Nombre del alumno 1}\\
\param{Legajo 1}\\
\param{Correo electrónico 1}\\
\param{/Firma del alumno 2, si corresponde}\\
\param{/Nombre del alumno 2, si corresponde}\\
\param{/Legajo 2, si corresponde}\\
\param{/Correo electrónico 2, si corresponde}\\
\param{Firma y aclaración Director}\\
\param{/Firma y aclaración Codirector, si corresponde}\\
\end{flushright}

