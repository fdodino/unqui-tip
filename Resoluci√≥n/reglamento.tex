
\anexo{
TECNICATURA UNIVERSITARIA EN PROGRAMACIÓN INFORMÁTICA \\
REGLAMENTO DE TRABAJO DE INSERCIÓN PROFESIONAL \\
DEPARTAMENTO DE CIENCIA Y TECNOLOGÍA
}

\capitulo{DEFINICIÓN, OBJETIVOS Y RECONOCIMIENTO}  

\articulo El Trabajo de Inserción Profesional (TIP) es una instancia de
caracter integrador que debe realizar y presentar todo alumno de la carrera
Tecnicatura Universitaria en Programación Informática para obtener el título
correspondiente.

\articulo{objetivos} El TIP tiene como objetivos:
\begin{itemize}
 \item Desarrollar e integrar los conocimientos y formación adquiridos a lo
 largo de la carrera.
 \item Promover el uso de la creatividad e iniciativa para afrontar con
 eficiencia el trabajo proyectado.
 \item Asegurar el empleo de una correcta metodología y criterio
 profesional para el logro del plan propuesto.
\end{itemize}

\articulo Los créditos y horas que se asignen al TIP serán los
correspondientes al plan de estudios de la carrera vigente al momento de su

\capitulo{CARACTERÍSTICAS, DURACIÓN Y ÁMBITO DE REALIZACIÓN DEL TRABAJO}

\articulo El TIP se podrá realizar en el ámbito de la Universidad Nacional de
Quilmes, en otras Universidades, en empresas, en Institutos de Investigación y
Desarrollo u otros organismos estatales o privados. 

\articulo El TIP será individual o en un grupo de a lo sumo dos estudiantes

\articulo El TIP podrá tener alguna de las tres siguientes modalidades:
\begin{enumerate}
\item Proyecto de desarrollo de un producto de software.
\item Trabajo de investigación teórica o de teoría aplicada.
\item Pasantía calificada en un sector relacionado al software en una
empresa u organización estatal o privada.
\end{enumerate}

El tema deberá estar dentro de la órbita de incumbencia del título de técnico y
podrá ser autocontenido o parte integrante de un desarrollo más complejo para ser
eventualmente continuado o completado por otro u otros TIPs.

\articulo La complejidad del TIP deberá permitir su realización en un
tiempo comprendido entre 100 (cien) y 180 (ciento ochenta) horas por estudiante de
acuerdo con el plan de estudios vigente. En los casos de trabajos compartidos, el
director debe certificar la equidad en la dedicación empleada y la concreción de los
objetivos mencionados en el artículo 2º por cada uno de los
integrantes. %TODO: HAcer una referencia en lugar de que el articulo 2 esté fijo

\articulo Para presentar la propuesta de realización del TIP, el o los
estudiantes deberán acreditar un 80\% del total de créditos necesarios para
acceder al título de Técnico. Para la defensa del TIP, no podrán adeudar más de
una asignatura

\articulo. El proceso de desarrollo del TIP se articulará en diversas etapas:
\begin{itemize}
 \item Presentación de la propuesta.
 \item Desarrollo del trabajo propiamente dicho.
 \item Presentación final
 \item Evaluación y defensa.
\end{itemize}

Cada una de estas etapas se regirá acorde con la normativa de los capítulos
ARTÍCULO 10°. La duración del TIP, desde la aprobación de la presentación hasta
su presentación final, está prevista en 6 (seis) meses. 

\capitulo{ DIRECCIÓN DEL TRABAJO DE INSERCIÓN PROFESIONAL}

\articulo. El TIP deberá contar con una dirección. La misma será llevada a
cabo por un Director y opcionalmente un Codirector si lo justifica el tema, plan de
trabajo u otras condiciones.

\articulo El Director del trabajo, y el Codirector, si existe,
deberán cumplir con al menos uno de los siguientes requisitos:

\begin{enumerate}
\item Ser profesor ordinario o interino en asignaturas de la Carrera, con al
menos dos cuatrimestres de docencia en los últimos dos años.
\item Ser profesor de otras carreras de la UNQ o de otras Universidades en
asignaturas afines a carreras de informática con al menos dos años de
antigüedad.
\item Ser investigador reconocido en temáticas afines al tema del trabajo.
\item Ser profesional de la industria informática con reconocida trayectoria en
temas afines al tema del trabajo.
\end{enumerate}

\articulo Cuando el director sea externo a la UNQ y no hubiera figura de
Codirector interno, deberá designarse un profesor ordinario o interino como tutor
académico de la institución.

\articulo El Director y Codirector sólo podrán dirigir y/o codirigir hasta 3
(tres) trabajos simultáneamente. 

\articulo Son funciones del Director y del Codirector si lo hubiere:
\begin{itemize}
 \item Decidir con el estudiante el tema y plan del TIP verificando que la
 propuesta puede desarrollarse en los tiempos establecidos en el
 artículo 7°.
 \item Supervisar el efectivo cumplimiento y dedicación del estudiante en
 sus actividades, estableciendo un método de seguimiento y
 manteniendo reuniones periódicas con el estudiante.
 \item Coordinar con el Director de Carrera el uso de los recursos de la
 UNQ verificando su disponibilidad.
 \item Controlar que se cumpla con el alcance del TIP propuesto en la
 presentación correspondiente.
 \item Ser plenamente responsable de todo lo presentado por el
 estudiante en relación al TIP.
 \item Informar periódicamente a la Dirección de Carrera el estado de
 avance del TP.
 \item Actuar como nexo entre el jurado evaluador y el estudiante.
 \item Oficiar como presentador del alumno y el TIP en la defensa oral del
 mismo.
 \item Participar activamente en la detección y solución de los problemas
 que pudieran surgir durante el desarrollo del trabajo.
\end{itemize}


\capitulo{PROPUESTA INICIAL}

\articulo Para comenzar el TIP, el estudiante deberá solicitar la aceptación
de su propuesta de Trabajo de Inserción Profesional mediante una presentación (ver
Anexo 1) firmada por el Director propuesto (y Codirector si lo hubiere) y el o los
estudiantes. En la misma deberá constar:

\begin{enumerate}[a.]
\item Tema del TIP.
\item Justificación e importancia del tema seleccionado.
\item Objetivos del trabajo.
\item Lugar donde se realizará el trabajo.
\item Plan de trabajo propuesto y su cronograma.
\item Fuentes iniciales de información y referencia.
\item Porcentaje de asignaturas aprobadas por cada uno de los estudiantes.
\item Foja académica de cada uno de los estudiantes.
\item Currículum vitae del director y codirector, si existe.
\end{enumerate}

\articulo Las propuestas de Trabajo de Inserción Profesional podrán ser
presentadas entre el 1 de Marzo y el 30 de Noviembre de cada año, excluyendo el
receso invernal.

\articulo La propuesta debidamente encarpetada debe ser entregada al
Director de la Carrera, quién podrá consultar a profesores de asignaturas afines al
tema de trabajo y al Director o Codirector propuesto, y luego, en un plazo no mayor a
15 días de recibida la propuesta, elevará un dictamen al Consejo, quién decidirá sobre
su aprobación. Tanto el Director de la Carrera como la Comisión de Investigación
Científica y Tecnológica podrán solicitar los ajustes que consideren necesarios para la
admisión de los TIPs. 

\articulo. Si el trabajo se realiza en un ámbito externo a la UNQ, deberá
contarse con el aval de la empresa o institución correspondiente. El Director de la
carrera elevará, de acuerdo a la normativa vigente en el Departamento, para los fines
de la cobertura por ART

\begin{itemize}
 \item una nota indicando apellido y nombre del/los alumno/s, DNI, fecha
 de nacimiento, actividad a desarrollar (en este caso, Trabajo de
 Inserción Profesional, con el tema y otros datos pertinentes) y lugar
 de desarrollo,
 \item certificación/aval del Director/Codirector externo (en caso de
 haberlo/s),
 \item certificación/Aval de Autoridad competente del ámbito externo (en
 caso de desarrollarse la actividad fuera de la UNQ).
\end{itemize}


\articulo El Consejo Departamental aprobará mediante resolución la
realización de los TIPs, especificando:
\begin{itemize}
 \item Apellido, nombre y número de legajo de los estudiantes.
 \item Fecha de inicio (que contará como el inicio de los plazos de
 realización del Art. 10°) y fecha estimada de finalización.
 \item Director (y Codirector o Tutor, si lo hubiere).
 \item Tema.
 \item Lugar de trabajo.
\end{itemize}

Se notificará debidamente al estudiante junto con las observaciones y/o cambios
propuestos para la aprobación del plan.

\capitulo{SEGUIMIENTO DEL TRABAJO}

\articulo En caso de incumplimiento por parte del o de los estudiantes, el
Director deberá notificar por escrito al Director de Carrera y podrá solicitar su
desvinculación del trabajo. El Director de Carrera deberá expedirse en un plazo
máximo de 15 días sobre la procedencia o improcedencia de dicha solicitud. En caso
de considerar procedente el planteo interpuesto por el Director, el Director de Carrera
deberá comunicar tal situación al Consejo Departamental. Si se ha aceptado la
desvinculación el Director de Carrera deberá comunicar el cese del desarrollo del TIP
y el estudiante deberá cumplir con lo estipulado en el Art. 25°.

\articulo En el caso de que el o los estudiantes consideren que el Director
o Codirector del trabajo incurre o incurren en incumplimiento de sus obligaciones.
deberá informar por nota al Director de la Carrera. El Director de Carrera deberá
expedirse en un plazo máximo de 15 días sobre la procedencia o improcedencia de
dicha solicitud. En caso de considerar procedente el planteo interpuesto por el
estudiante, el Director de Carrera deberá comunicar tal situación al Consejo
Departamental y presentar la propuesta de un nuevo Director o Codirector. 

\articulo El Director de la Carrera podrá autorizar una prórroga de hasta 6
meses ante un pedido escrito y debidamente fundamentado del estudiante y avalado
por el Director del TIP (y el Codirector si lo hubiere). Excepcionalmente el Consejo
Departamental podrá otorgar una extensión no mayor a 6 meses adicionales ante
solicitud fundada presentada por el estudiante y avalada por el Director del TIP (y el
Codirector si lo hubiere) y el Director de la Carrera. 

\articulo El estudiante podrá solicitar al Director de la Carrera una
suspensión de plazos mediante nota debidamente justificada y avalada por el Director
del TIP (y el Codirector si lo hubiere). De serle concedida se interrumpirá la cuenta de
los tiempos establecidos en el Art. 10° (incluyendo las prórrogas mencionadas en el
artículo 23°). El Director de la Carrera deberá informar al Consejo Departamental
acerca de las suspensiones de plazos. En el caso de una suspensión mayor a 3 (tres)
meses el estudiante deberá dejar disponible para otros usos todo material, insumo,
lugar físico o equipamiento provisto por la UNQ.

\articulo El TIP se dará por finalizado si no se hubiese efectuado la
defensa transcurridos los plazos establecidos en el Art. 10° (incluyendo las prórrogas
mencionadas en el artículo 23°). En este caso el estudiante deberá desocupar el lugar
físico y devolver el equipamiento provisto por la UNQ y restituir todo material o insumo
utilizado en el proceso inconcluso.

\capitulo{EVALUACIÓN}

\articulo La culminación del TIP implica su presentación por escrito y su
defensa oral y pública.

\articulo Para la presentación escrita y la defensa oral del TIP se formará
un Jurado evaluador, propuesto por el Director del trabajo y conformado por el Director

\articulo El jurado constará de tres integrantes, al menos dos de los cuales
serán profesores pertenecientes a la UNQ y al menos uno de ellos a la Carrera de
Tecnicatura Universitaria en Programación Informática. El miembro restante podrá ser
un profesional de reconocida trayectoria y ser externo a la Universidad Nacional de
Quilmes. El Director y/o Codirector del TIP y el Director de la Carrera no podrán ser
miembros del Jurado. Todos los integrantes del Jurado deben cumplir con al menos
una de las condiciones requeridas en el Art. 12°.

\articulo El o los estudiantes y el Director del TIP (y Codirector si lo
hubiere) podrán sugerir, por nota escrita al Director de la Carrera, potenciales
integrantes del jurado.

\articulo El o los estudiantes en conjunto con el Director del TIP (y
Codirector si lo hubiere) podrán recusar o pedir el cambio de Jurado con la
fundamentación adecuada mediante nota al Director de la Carrera quien la elevará al
Consejo Departamental.

\articulo En el caso de los jurados externos a la UNQ se deberá contar
con una aceptación de la tarea por parte de los mismos. En el caso de los jurados
internos de la UNQ podrán excusarse mediante justificación debidamente fundada. 

\capitulo{PRESENTACIÓN DEL INFORME}

\articulo La presentación por escrito del TIP deberá ser realizada teniendo
en cuenta las recomendaciones que figuran en el Anexo II, acompañada de una nota
según el modelo del Anexo III. Se entregará un ejemplar impreso del informe y una
copia electrónica del mismo al Director de la Carrera.

\articulo En primera instancia se remitirá al Jurado el informe final del TIP
en formato electrónico para la elaboración del dictamen inicial. El Jurado tendrá un
plazo no superior a los 30 (treinta) días corridos a partir de la recepción de la versión
electrónica, para realizar una evaluación preliminar y comunicar al Director de la
Carrera su dictamen inicial, fundamentado y por escrito. El jurado podrá expedirse en
tres alternativas:

\begin{enumerate}[a.]
\item Habilitado para la defensa oral.
\item Devuelto con observaciones.
\item Rechazado con dictamen fundado.
\end{enumerate}

\articulo En el caso del Art. 33°, inciso a., los estudiantes estarán
habilitados para la presentación oral. 

\articulo En el caso del Art. 33°, inciso b., los estudiantes podrán rehacer
la presentación una única vez, con las consideraciones que haya sugerido el Jurado.
En función de las correcciones realizadas, el Jurado determinará un plazo de entre
uno y tres meses para la realización de las modificaciones y solicitar un nuevo
dictamen inicial.

\articulo En el caso del Art. 33°, inciso c., los estudiantes entrarán en las
condiciones pautadas en el Art. 25°. 

\capitulo{DEFENSA ORAL}

\articulo Una vez habilitados para la defensa oral, los estudiantes
entregarán 2 (dos) ejemplares impresos de la versión definitiva a la Dirección de la
Carrera, encuadernados en formato A4 y debidamente numerados.

\articulo El o los estudiantes y el Director (y Codirector si lo hubiere)
coordinarán la fecha de defensa oral con los miembros del Jurado, dentro de los 30
días a partir de la entrega de la versión final.

\articulo La Dirección de la Carrera se hará cargo de las gestiones
necesarias que habiliten la defensa oral y pública en el día prefijado.

\articulo La defensa oral y pública constará de una exposición que incluirá
los aspectos más sobresalientes del trabajo realizado. Tendrá una duración mínima de
30 minutos y máxima de 45 minutos, y una sesión consecutiva de preguntas y
consideraciones de los miembros del tribunal.

\articulo En la defensa oral y pública el dictamen del jurado, con el criterio
de la mayoría simple, podrá resultar: 
\begin{enumerate}[a.]
\item Aprobado con dictamen fundado.
\item Rechazado con dictamen fundado.
\end{enumerate}

\capitulo{ARCHIVO Y DIFUSIÓN}

\articulo Una vez aprobado el TIP, un ejemplar será destinado a la
Biblioteca general de la Universidad Nacional de Quilmes y otro quedará en la
Dirección de la Carrera. 

\articulo Los eventuales aspectos legales referidos a la propiedad
intelectual que pueden derivar del desarrollo de los Trabajos de Inserción Profesional
serán resueltos de acuerdo a los reglamentos y resoluciones vigentes en la
Universidad Nacional de Quilmes.

\capitulo{VÍA DE EXCEPCIÓN}

\articulo Toda excepción a este reglamento deberá ser analizada en
primera instancia por el Director de Carrera y aprobada por el Consejo Departamental.

\resetAnexosCounter
\anexo{NOTA MODELO del alumno para aprobación de la propuesta de trabajo de
inserción profesional}

OBSERVACIÓN: los campos consignados como \textless...\textgreater deben ser reemplazados
por el contenido correspondiente. Aquellos que ofreciesen alternativas
 \textless.../...\textgreater deberán reemplazarse por la alternativa
 correspondiente.

\begin{flushright}
Bernal, \param{día} de \param{mes} de \param{año}\end{flushright}

Al Director de la Carrera de Tecnicatura Universitaria en Programación Informática del
Departamento de Ciencia y Tecnología de la
Universidad Nacional de Quilmes,\\
\param{Título, Nombre y Apellido del Director de Carrera}

\hrulefill

Asunto: Propuesta de trabajo de inserción profesional

\param{Me dirijo/Nos dirigimos} a Usted para presentar \param{mi/nuestra}
propuesta de Trabajo de Inserción Profesional de la carrera de Tecnicatura
Universitaria en Programación Informática. El tema del trabajo es: \param{Tema
del trabajo}, y será dirigido por \param{Título, Nombre y Apellido del Director
del Trabajo>} y \param{codirigido por: \param{Título, Nombre y Apellido del
Codirector del Trabajo} si corresponde}.

Adjunto con esta nota la justificación e importancia del tema
seleccionado, los objetivos del trabajo, el lugar donde se ha de realizar el
mismo, el plan de trabajo y su cronograma así como las fuentes iniciales de
información y referencia, según el reglamento vigente. \param{Poseo/Poseemos}
aprobadas el \param{Porcentaje de aprobadas}\% \param{y el \param{Porcentaje de
aprobadas del alumno 2}, si corresponde} de las materias de la carrera (se
\param{adjunta certificado analítico/adjuntan certificados analíticos}).

Sin otro particular \param{saluda/saludan} atentamente.\\ \\
\param{Firma, aclaración y Nro Legajo de alumno 1}\\
\param{Firma, aclaración y Nro Legajo de alumno 2, si corresponde}\\
\param{Firma y aclaración Director}\\
\param{Firma y aclaración Codirector, si corresponde}\\

\anexo{PRESENTACIÓN ESCRITA DEL TRABAJO DE INSERCIÓN PROFESIONAL}

Primera Parte: Forma.
\begin{enumerate}
\item Hojas A4, numeradas, letra tamaño 12pt.
\item Trabajo encuadernado o anillado.
\item Modelo de portada (centrado en la hoja).
\begin{center}
UNIVERSIDAD NACIONAL DE QUILMES\\
Departamento de Ciencia y Tecnología\\
Tecnicatura Universitaria en Programación Informática\\

\param{TITULO DEL TRABAJO}\\
\param{NOMBRE DEL ALUMNO O ALUMNOS}\\
\param{DIRECTOR (Y CODIRECTOR) DEL PROYECTO}\\
\param{FECHA}\\
\end{center}
\end{enumerate}

Segunda Parte: Contenido.
La estructura y contenido del trabajo de inserción profesional dependerán de la
modalidad del trabajo. En todos los casos de deberá incluir:
\begin{enumerate}[a.]
\item Resumen.
\item Índice.
\item Introducción e importancia del tema.
\item Objetivos.
\item Marco teórico.
\item Conclusiones.
\item Referencias bibliográficas.
\end{enumerate}
\anexo{

NOTA MODELO para la presentación del informe final del Trabajo de Inserción}

OBSERVACIÓN: los campos consignados como \textless...\textgreater deben ser
reemplazados por el contenido correspondiente. Aquellos que ofreciesen
alternativas \textless.../...\textgreater deberán reemplazarse por la alternativa correspondiente.
\begin{flushright}
Bernal, <día> de <mes> de <año>
\end{flushright}

Al Director de la Carrera de Tecnicatura Universitaria en Programación Informática del
Departamento de Ciencia y Tecnología de la
Universidad Nacional de Quilmes,

\param{Título, Nombre y Apellido del Director de Carrera}

\hrulefill

\param{Tengo/Tenemos} el agrado de \param{dirigirme/dirigirnos} a
usted para hacerle llegar 2 (dos) copias del informe escrito del Trabajo de Inserción
Profesional titulado \param{Título del trabajo de inserción profesional}, que
fuera dirigido por \param{Título, Nombre y Apellido del Director del Trabajo}
\param{/y codirigido por \param{Título, Nombre
y Apellido del Codirector del Trabajo}, si corresponde}

Se propone como posibles jurados titulares a \param{Título,
Nombre y Apellido del Jurado 1}, \param{Título, Nombre y Apellido del Jurado 2}
y \param{Título, Nombre y Apellido del Jurado 3} y como posibles jurados
suplentes a \param{Título, Nombre
y Apellido del Jurado 4} y \param{Título, Nombre y Apellido del Jurado 5}.
del jurado, y de su expedición sobre el trabajo. Cumplido esto, coordinaremos la fecha
de defensa oral.

\param{Quedo/Quedamos} a la espera de la designación oficial

Sin otro particular lo saluda atentamente\\ \\
\param{Firma del alumno 1}\\
\param{Nombre del alumno 1}\\
\param{Legajo 1}\\
\param{Correo electrónico 1}\\
\param{/Firma del alumno 2, si corresponde}\\
\param{/Nombre del alumno 2, si corresponde}\\
\param{/Legajo 2, si corresponde}\\
\param{/Correo electrónico 2, si corresponde}\\
\param{Firma y aclaración Director}\\
\param{/Firma y aclaración Codirector, si corresponde}\\



